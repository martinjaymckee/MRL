\documentclass[letterpaper,10pt]{article}
\usepackage[utf8]{inputenc}

%opening
\title{MRL Control JSON Protocol}
\author{Martin Jay McKee}
\date{}

\begin{document}

\maketitle

\begin{abstract}
This document outlines the format of the JSON messages that are passed between the new rovers and the control system.  Note, these formats are preliminary and are subject to change.
\end{abstract}

\section{Frame Formats}
  IT WOULD BE GOOD TO ADD FRAMES TO HANDLE THE ROVER LOCATION SO THAT IT IS POSSIBLE TO DISPLAY A MAP ON THE CLIENT.  WHILE SUCH A MAP IS A NICE FEATURE, HOWEVER, IT IS HIGHLY LIKELY TO DRIFT GIVEN NO ABSOLUTE POSITIONING SYSTEM AND, AS SUCH, IT SHOULD BE RESET FROM TIME TO TIME.  THIS IS AN ADVANCED FEATURE AND NOT AT ALL IMPORTANT FOR A FIRST RUN OF THE SYSTEM.  STILL, THIS CAN POTENTIALLY BE BASED, IN PART, UPON THE ROVER STATUS MESSAGES.
  
  \subsection{Error Format}
    An error may be sent as a subframe in status frames.  It is never sent alone and, therefore, is missing some of the basic fields of the remaining frames.  An error frame contains information to describe the error as well as to command a response to it.
    
    \begin{tabular}{ccc}
      \textbf{Field} & \textbf{Type} & \textbf{Description}\\\hline
      name & String & This field identifies the name of the error.  It is upper-case and contains no whitespace.\\
      level & String(enum) & This is a string from the list [Debug, Info, Warning, Error, Fatal].\\
      message & String & This is a human-readable description of the error for display on the client.\\
      action & List of Frames & This defines the actions to be taken by the client in response to the error.\\
    \end{tabular}
    
  \subsection{Basic Format}
    Each frame has a basic set of fields.  These fields determine the type of the frame as well as give it a unique idenitifer.
      
    \begin{tabular}{ccc}
      \textbf{Field} & \textbf{Type} & \textbf{Description}\\\hline
      type & String & This field identifies the type of frame.  It is a lower-case string without whitespace.\\
      id & Unsigned Integer & This field is a unique id for the frame\\
    \end{tabular}

  \subsection{Server Frames}
    These are frames that are sent out from the server to the client.  TODO: ADD FRAMES WHICH CAN RETURN MORE ROVER STATUS INFORMATION.  FOR INSTANCE, IT WOULD BE NICE TO HAVE THE ABILITY TO RETURN OBSTACLE DATA TO THE CONTROL SYSTEM.
    
    \subsubsection{Rover Config (rover)}
      This frame is required to correctly configure the control system.  It contains information about the physical characteristics of the rover which are used to set distance and velocity limits within the interface.  This frame will be sent when a connection is established or when the values need to be updated.
      
      \begin{tabular}{ccc}
	\textbf{Field} & \textbf{Type} & \textbf{Description}\\\hline
	wheelbase & Positive Float & Length between the center of the front and back wheels (m)\\
	track & Positive Float & Distance between the center of the left and right wheels (m)\\
	vlimit & Positive Float & Maximum velocity (m/s)\\
	dlimit & Positive Float & Maximum distance to travel in a single command (m)\\
	alimit & Positive Float & Maximum linear acceleration (m/s2)\\
	wlimit & Positive Float & Maximum angular velocity (deg/s) ?? CALCULATED\\	
	wdotlimit & Positive Float & Maximum angular acceleration (deg/s2) ?? CALCULATED\\
	pmin & Float [0, 100] & Percentage of minimum ``power''\\
	pmax & Float [0, 100] & Percentage of maximum ''power``\\
      \end{tabular}

    \subsubsection{Client Config (client)}
      This frame is required to correctly configure the control system client.  It contains information about the allowable control messages and control interface formats.  This frame will be sent when a connection is established or when the values need to be updated.
      
      \begin{tabular}{ccc}
	\textbf{Field} & \textbf{Type} & \textbf{Description}\\\hline
	cid & Unsigned Integer & The client id value for the control client to use\\
	pivotable & Boolean & Set to true if the rover supports in-place pivot\\
	advancedmode & Boolean & Set to true if controlling the rover in advanced mode is supported\\
      \end{tabular}
      
    \subsubsection{Command Status (cmdstatus)}
      This status frame returns the current progress of a command sent to the rover.  This command is broadcast to all connected clients.
      
      \begin{tabular}{ccc}
	\textbf{Field} & \textbf{Type} & \textbf{Description}\\\hline
	src.type & String & This is the message type of the source command whose status is being returned\\
	src.cid & Unsigned Integer & This is the client id of the source command's originating client\\
	src.id & Unsigned Integer & This is the id value of the source command whose status is being returned\\
	progress & Float [0, 100] & The approximate percentage of command completion\\
	error & Dictionary/nil & Either empty (nil) or a dictionary describing the error\\
      \end{tabular}

    \subsubsection{Rover Status (roverstatus)}
      This status returns the current state of the rover: orientation, position, system status, etc.  Note that although the attitude and position will be returned in three-dimensional format, the initial implementation of the rover systems can be strictly two-dimensional.  That would constrain the roll and pitch to zero degrees while constraining the z axis to 0.  NOTE: IS IT BETTER TO RETURN ATTITUDE AS EULER ANGLES OR A QUATERNION???
      
      \begin{tabular}{ccc}
	\textbf{Field} & \textbf{Type} & \textbf{Description}\\\hline
	attitude & 3d Float Vector & The rover angles, [roll, pitch, yaw] (deg)\\
	position & 3d Float Vector & The rover offset from initial, [x, y, z] (m)\\
	traveled & Positive Float & Total distance traveled (m)\\
	battery & Float [0, 100] & The approximate percentage of battery remaining\\
	error & Dictionary/nil & Either empty (nil) or a dictionary describing the error\\
      \end{tabular}
        
  \subsection{Client Frames}
    \subsubsection{Set (set)}
      This frame allows the setting of values in the control system.  It requires a user that is logged in as an administrator.
      
      \begin{tabular}{ccc}
	\textbf{Field} & \textbf{Type} & \textbf{Description}\\\hline
	cid & Unsigned Integer & The client id\\
	name & String & Name of the value to be set\\
	value & Variant & The value to set the variable to\\
      \end{tabular}
      
      NOTE: MORE DESCRIPTION OF THE AVAILABLE VARIABLES AND THEIR TYPES NEEDS TO BE INCLUDED.  IT PROBABLY MAKES SENSE TO BE ABLE TO GET VALUES AS WELL AS TO LIST THE AVAILABLE VALUES.  THE GET FRAME CAN SIMPLY REQUIRE A PATTERN (I.E. A REGEX) WHICH IS USED TO CREATE THE LIST OF VARIABLE VALUES TO RETURN.  THIS IS USEFUL AS BOTH AN EXPLICIT GET FUNCTIONALITY AND AS A LIST OPERATION.  BECAUSE THE OPERATIONS ARE ASYNCHRONOUS, IT COULD REQUIRE SOME INTERESTING PROGRAMMING ON THE CLIENT SIDE, HOWEVER, FOR OPERATIONS WHICH ARE SUPPOSED TO LOOK SYNCHRONOUS.
  
    \subsubsection{Move}
      This frame provides a single move command for a rover.  The command can define either a move along a circular arc or an in-place rotation.  Limits on both linear and angular accelerations are included as well as the total arc distance to be traveled, the target velocity and the angular displacement.
      
      \begin{tabular}{ccc}
	\textbf{Field} & \textbf{Type} & \textbf{Description}\\\hline
	cid & Unsigned Integer & The client id\\
	v & Float & Linear velocity at which to travel\\
	alimit & Positive Float & Linear acceleration limit\\
	d & Positive Float & Linear distance to travel\\
	theta & Float & Angular displacement of travel\\
	wlimit & Positive Float & Angular velocity limit\\
	wdotlimit & Positive Float & Angular acceleration limit\\
      \end{tabular}
      
      
\end{document}
